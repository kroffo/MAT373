\documentclass{scrartcl}
\usepackage[margin=1in]{geometry}
\usepackage{amsmath,amssymb,commath}
\setkomafont{disposition}{\normalfont\bfseries}

\title{Number Theory}
\subtitle{Homework 4}
\author{Kenny Roffo}
\date{Due February 22, 2016}

\begin{document}
\maketitle

\textbf{1)} Show that 3, 5, 7 is the only prime triplet.\\

We begin by seeing that 9 plus a multiple of 6 is never prime.
\begin{align*}
  9 + 6n &= 3(3) + 3\cdot2(n)\\
  &= 3\left(3 + 2n\right)
\end{align*}
Now we examine the odd triplets. Let $P(n)$ be the statement that one of $n, n+2, n+4$ is 9 plus a multiple of 6. $P(5)$ is true since 5, 7, 9 contains 9 and $9 = 9 + 6(0)$. Now let $k$ be an odd integer greater than or equal to 5 and assume $P(k)$ to be true. Then $k, k+2$ or $k+4$ is 9 plus a multiple of 6. If it is $k$, then $k=9+6(m)$ for some integer $m$ and the next prime triplet contains $k+6 = (k+2)+4$. We see $$k+6 = 9 + 6(m) + 6 = 9 + 6(m+1)$$ so $(k+2)+4$ is 9 plus a multiple of 6, so $P(k+2)$ is true. Now if it is $k+2$ or $k+4$ is 9 plus a multiple of 6, then the next prime triplet (which contains both $k+2$ and $k+4$) contains a number which is 9 plus a multiple of 6 thus $P(k+2)$ is true. Thus we have shown that if $P(k)$ is true, then $P(k+2)$ is true. Therefore by the principle of mathematical induction $P(n)$ is true for all odd integers greater than or equal to 5. Thus there exist no prime triplets that begin with a number greater than or equal to 5. Since no number below 2 is prime, and 2, 4, 6 is not a prime triplet, we have shown that 3, 5, 7 is the only prime triplet.\pagebreak

\textbf{2)} Let $p, p+2$ be twin primes with $p>3$. Prove the sum is divisible by 12.\\

All numbers can be written in one of the following forms:
\begin{align*}
  6n+0 &= 2(3n+0)\\
  6n+1\\
  6n+2 &= 2(3n+1)\\
  6n+3 &= 3(2n+1)\\
  6n+4 &= 2(3n+2)\\
  6n+5
\end{align*}
As shown above, no number of the forms $6n+0$, $6n+2$, $6n+3$ and $6n+4$ can be prime, because they have divisors other than 1 and themselves (and thus more than 2 divisors). Note also, that a number of the form $6n+5$ may also be written (for a different $n$) as $6n-1$. Thus all primes are of the form either $6n-1$ or $6n+1$. It is obvious that a pair of twin primes cannot both be of the same form, thus any pair of twin primes has the form $6n-1$ and $6n+1$. Summing twin primes, we see $$ (6n-1)+(6n+1) = 12n $$ thus for any pair of twin primes, 12 is a divisor of their sum.\\

\textbf{3)} Find all perfect squares of the form $17p+1$ where $p$ is prime.\\

Let $p$ be a prime such that $17p+1=x^2$ where $x\in\mathbb{Z}$. Manipulating this equality, we see
\begin{align*}
  &17p+1 = x^2\\
  \implies& 17p = x^2 - 1\\
  \implies& 17p = (x-1)(x+1)
\end{align*}
We now have a product of two primes equal to a product of two integers. If the right side contained a non-prime integer, then it could be rewritten as a product of two or more primes. In this case, we would have two distinct prime factorizations equal to each other, a contradiction. Thus $x-1$ and $x+1$ must be prime, and furthermore one of the two must be 17. If $x-1=17$, then $x+1=19$, which is prime, so $p=19$ is a solution. If $17=x+1$, then $p=x-1=15$ is not prime, thus is not a solution. Therefore $p=19$ is the only possible prime solution to the equation $17p+1=x^2$.\\

\textbf{4)} Show that no number of the form $8^n+1$ is prime where $n$ is a positive integer.\\

This becomes simple when we realize that this is a sum of perfect cubes:
\begin{align*}
  8^n + 1 &= 2^{3n} + 1^3\\
  &= \left(2^n + 1\right)\left(2^{2n} - 2^n + 1\right)
\end{align*}
By the closure laws, both polynomials are integers, and of course $1 < 2^n + 1 < 8^n + 1$, so $8^n+1$ has more than two positive divisors, thus it is not prime.
\end{document}
