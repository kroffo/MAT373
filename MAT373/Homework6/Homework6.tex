\documentclass{scrartcl}
\usepackage[margin=1in]{geometry}
\usepackage{amsmath,amssymb,commath}
\setkomafont{disposition}{\normalfont\bfseries}
\renewcommand{\mod}[1]{\text{ (mod $#1$)}}

\title{Number Theory}
\subtitle{Homework 6}
\author{Kenny Roffo}
\date{Due March 7, 2016}

\begin{document}
\maketitle

\textbf{1)} If gcd$(a,30)=1$, show that 60 divides $a^4+59$.\\
\begin{tiny}[I know there must be a better way to do this, but after many hours this was the best I could come up with]\end{tiny}

Let $a$ be an integer such that $(a,30)=1$. Then none of 2, 3 and 5 may divide $a$. Thus $a \equiv x \mod{60}$ where $0 \le x < 60$ and none of 2, 3 and 5 divide $x$. We see $a^4 \equiv x^4 \mod{60}$, so we examine the value of $x^4 \mod{60}$ for all possible values of $x$:

\begin{flalign*}
x=7: && 7^4 \equiv 2401 \equiv 60(40)+1 &\equiv 1 \mod{60}&\\
x=11: && 11^4 \equiv 14641 &\equiv 1 \mod{60}&\\
x=13: && 13^4 &\equiv 1 \mod{60}&\\
x=17: && 17^4 &\equiv 1 \mod{60}&\\
x=19: && 19^4 &\equiv 1 \mod{60}&\\
x=23: && 23^4 &\equiv 1 \mod{60}&\\
x=29: && 29^4 &\equiv 1 \mod{60}&\\
x=31: && 31^4 &\equiv 1 \mod{60}&\\
x=37: && 37^4 &\equiv 1 \mod{60}&\\
x=41: && 41^4 &\equiv 1 \mod{60}&\\
x=43: && 43^4 &\equiv 1 \mod{60}&\\
x=47: && 47^4 &\equiv 1 \mod{60}&\\
x=49: && 49^4 &\equiv 1 \mod{60}&\\
x=53: && 53^4 &\equiv 1 \mod{60}&\\
x=59: && 59^4 &\equiv 1 \mod{60}&\\
\end{flalign*}

So no matter what, $a^4 \equiv 1 \mod{60}$. But also $a^4 \equiv 1 \equiv -59 \mod{60}$ which implies $60x | a^4 + 59$.
\pagebreak

\textbf{2)} If $7 \nmid a$, prove that either $a^3+1$ or $a^3-1$ is divisible by 7.\\

Let $a$ be an integer which is not divisible by 7. Then $a \equiv x \mod{7}$ where $x\in\{1,2,3,4,5,6\}$. Now we examine the different cases for $x$:

\begin{flalign*}
x=1: && a^3 \equiv 1^3 &\equiv 1 \mod{7}&\\
x=2: && a^3 \equiv 2^3 \equiv 8 &\equiv 1 \mod{7}&\\
x=3: && a^3 \equiv 3^3 \equiv 27 &\equiv -1 \mod{7}&\\
x=4: && a^3 \equiv 4^3 \equiv 64 &\equiv 1 \mod{7}&\\
x=5: && a^3 \equiv 5^3 \equiv 125 &\equiv -1 \mod{7}&\\
x=6: && a^3 \equiv 6^3 \equiv 216 &\equiv -1 \mod{7}
\end{flalign*}

No matter what $x$ is, it is apparent that $a^3$ is congruent to either 1 or -1 (mod 7), which means by definition that 7 divides either $a^3+1$ or $a^3-1$.\\
 
\textbf{3)} The three most recent appearances of Halley's comet were in the years 1835, 1910, and 1986; the next occurrence will be in 2061. Prove that $$1835^{1910}+1986^{2061}\equiv0 \mod{7}$$

We begin by noting that $1835\equiv1\mod{7}$ and $1986\equiv5\mod{7}$. This means
$$1835^{1910}+1986^{2061} \equiv 1^{1910}+5^{2061} \equiv 1 + 5^{2061}\mod{7}$$
Now we examine $5^{2061}$. We know by Fermat's Little Theorem that $5^6\equiv1\mod{7}$, and $(5^6)^n\equiv1^n\equiv1\mod{7}$. Applying this, we see
$$5^{2061} = 5^{6(343)+3} = (5^6)^{343}5^3 \equiv (1)^{343}5^3 \equiv 125 \equiv 6 \mod{7}$$
And applying this result we have
$$1835^{1910}+1986^{2061} \equiv 1 + 6 \equiv 7 \equiv 0 \mod{7}$$
And we are done.\\

\textbf{4)} If $p$ and $q$ are distinct primes, prove that $p^{q-1}+q^{p-1}\equiv1 \mod{pq}$\\

Let $p$ and $q$ be distinct primes. By Fermat's Little Theorem we know $p^{q-1}\equiv1\mod{q}$ and $q^{p-1}\equiv1\mod{p}$ which means there exist integers $x$ and $y$ such that $$qx = p^{q-1}-1 \text{\hspace{2in}} py = q^{p-1}-1$$ Multiplying and manipulating we see
\begin{flalign*}
  && qxpy &= \left(p^{q-1}-1\right)\left(q^{p-1}-1\right)&\\
  \implies && qxpy &= p^{q-1}q^{p-1} - p^{q-1} - q^{p-1} + 1&\\
  \implies && p^{q-1}q^{p-1} - qxpy &= p^{q-1} + q^{p-1} - 1&\\
  \implies && pq\left(p^{q-2}q^{p-2} - xy\right) &= \left(p^{q-1} + q^{p-1}\right) - 1
\end{flalign*}
This gives us the result we want as long as $p^{q-2}q^{p-2} - xy$ is an integer. Since $q$ and $p$ are prime, they are at least 2, thus the exponents are both at the very least 0, so it is true that $p^{q-2}q^{p-2} - xy$ is an integer. So $pq$ divides $\left(p^{q-1} + q^{p-1}\right) - 1$, which means $$p^{q-1} + q^{p-1} \equiv 1 \mod{pq}$$
\end{document}
