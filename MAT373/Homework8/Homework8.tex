\documentclass{scrartcl}
\usepackage[margin=1in]{geometry}
\usepackage{amsmath,amssymb,commath}
\setkomafont{disposition}{\normalfont\bfseries}
\renewcommand{\mod}[1]{\text{ (mod $#1$)}}

\title{Number Theory}
\subtitle{Homework 8}
\author{Kenny Roffo}
\date{Due April 4, 2016}

\begin{document}
\maketitle

\textbf{1a)} Let $p$ be an odd prime. Show that the Diophantine equation $$x^2 + py + a = 0 \text{\hspace{1in}} (a,p)=1$$ has an integral solution if and only if $(-a/p)=1$.

According to the quadratic formula, if the above equation has solutions $x$ and $y$, then the following holds:
$$x = \frac{-0 \pm \sqrt{0^2-4(1)(py+a)}}{2(1)} = \pm\sqrt{-py-a}$$
That is, if the above equation has integer solutions $x$ and $y$, then $x = \pm\sqrt{-py-a}$ (where $x$ and $y$ are integers).\\

$(\Rightarrow):$ Assume $x^2 + py + a = 0$ has integer solutions. Then $x = \pm\sqrt{-py-a}$. Thus $-py-a$ must be a perfect square, so there exists an integer $n$ such that $$n^2 = -py-a$$ This implies

\begin{flalign*}
  && &p(-y) = n^2 + a&\\
  \implies && &p | (n^2 + a)&\\
  \implies && &n^2 \equiv -a \mod{p}&\\
  \implies && &(-a/p) = 1
\end{flalign*}

$(\Leftarrow)$ Now assume $(-a/p) = 1$. Then there exists an integer $n$ such that

\begin{flalign*}
  && &(-a/p) = 1&&\\
  \implies && &n^2 \equiv -a \mod{p}&\\
  \implies && &p | (n^2 + a)&\\
  \implies && &pc = n^2 + a & (c\in\mathbb{Z})\\
  \implies && &n^2 = -p(-c) - a\\
  \implies && &n^2 = -py - a & (y=-c)\\
\end{flalign*}

That is, there exists an integer $y$ such that $-py - a$ is a perfect square. Therefore $x = \sqrt{-py-a}$ is an integer, and so the equation
$$x^2 + py + a = 0$$ has integer solutions.\\


\textbf{1b)} Determine whether $x^2 + 7y - 2 = 0$ has a solution in the integers.\\

By 1a, the given equation has integer solutions if and only if $(2/7) = 1$. It was shown in class that $(2/p) = 1$ where $p$ is prime if $p\equiv\pm 1 \mod{8}$. $7\equiv-1\mod{8}$, therefore indeed $(2/7) = 1$, so the given equation does have an integer solution.\\


\textbf{2a)} If $p$ is an odd prime and $(ab,p) = 1$, prove that at least one of $a, b$ or $ab$ is a quadratic residue of $p$.\\

It was shown in class that if $p$ is an odd prime, and $a,b$ are integers such that $p\nmid a$ and $p\nmid b$ then $(ab/p) = (a/p)(b/p)$. Since $(ab,p)=1$, $p\nmid ab$, thus $p \nmid a$ and $p \nmid b$, since $p$ is prime (If $p$ divided either one, then it would have to be the case that $p$ divided their product). Thus the result discussed in class applies. Now, if either $(a/p)$ or $(b/p)$ is 1, then the result follows. If not, then we have $$(ab/p) = (a/p)(b/p) = (-1)(-1) = 1$$ and the result still follows. Therefore, at least one of $(a/p), (b/p), (ab/p)$ must be 1.\\


\textbf{2b)} Given a prime $p$, show that, for some choice of $n>0$, $p$ divides $$(n^2-2)(n^2-3)(n^2-6)$$

Consider $n=p+1$. Then $$n^2-2 = (p+1)^2 - 2 = p^2 + 2p = p(p + 2)$$ So $$p | (n^2-2)(n^2-3)(n^2-6)$$ if $$p | p(p+1)(n^2-3)(n^2-6)$$ which is obviously true.\\


\textbf{3)} Determine whether the following quadratic congruence is solvable: $$x^2 \equiv 219 \mod{419}$$

The above congruence is solvable if its corresponding Legendre symbol, $(219/419)$, is 1. We use the corollary to the Law of Quadratic Reciprocity to find this value (note that 419 is prime):

\begin{flalign*}
  && (219/419) &= (73/419)(3/419)\\
  && &= (419/73)(3/419) & \text{By LQR}\\
  && &= (54/73)(3/419)\\
  && &= (54/73)(1) &\text{since }419\equiv-1\mod{12}\text{ by lemma presented in class}\\
  && &= (3/73)^3(2/73)\\
  && &= (1)^3(1) &\text{since }73\equiv1\mod{12}\text{ and }73\equiv1\mod{4}\\
  && &= 1
\end{flalign*}

So the given quadratic congruence is solvable.\pagebreak

\textbf{4)} Let $p,q$ be twin primes such that $x^2\equiv p\mod{q}$.  Prove $x^2\equiv q\mod{p}$ is solvable.\\

Since $p$ and $q$ are twin primes, $q$ is either $p + 2$ or $p - 2$. Note that all primes are of the form either $4k+1$ or $4k+3$. Whichever form $p$ has, $q$ must be of the other form. By the Law of Quadratic Reciprocity, we know
\begin{align*}
  (p/q)(q/p) &= (-1)^{(\frac{p-1}{2})(\frac{q-1}{2})}\\
  &= (-1)^{(\frac{4k+1-1}{2})(\frac{4k+3-1}{2})} \text{\hspace{1in} Note the order may have switched here}\\
  &= (-1)^{(2k)(2k+1)}\\
  &= 1
\end{align*}

That is, $(p/q)(q/p) = 1$ and since $(p/q) = 1$, this implies $(q/p)=1$. Therefore, $x^2\equiv q\mod{p}$ is solvable.

\end{document}
